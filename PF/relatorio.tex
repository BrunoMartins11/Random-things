\documentclass[a4paper]{article}

\usepackage[utf8]{inputenc}
\usepackage[portuges]{babel}
\usepackage{a4wide}

\title{Projeto de Laboratórios de Informática 1\\Grupo 163}
\author{Bruno Manuel Chaves Martins (A80410) \and Autor2 (número)}
\date{\today}

\begin{document}

\maketitle

\begin{abstract}
  Este documento apresenta o relatório do projeto de Laboratórios de Informática
  1 (LI1), da Licenciatura em Engenharia Informática da Universidade
  do Minho.

  O \emph{resumo} dum relatório deve sumarizar, em não mais que dois
  parágrafos, o trabalho desenvolvido e o relatório em questão,
  incluindo o problema a resolver, os resultados obtidos e as
  conclusões.
\end{abstract}

\tableofcontents

\section{Introdução}
\label{sec:intro}

Este documento apresenta o relatório do trabalho da UC de 
Laboratórios de Informatica do primeiro ano do Mestrado Integrado 
em Engenharia Informática da Universidade do Minho, onde foi proposto 
a reprodução do jogo Bomberman.

Este desafio foi-nos posto para a aplicação de conceitos teóricos 
lecionados na UC de Programação Funcional com o intuito de utilizar a
linguagem de programação \emph{Haskell}e desta forma melhorar a utilização 
da mesma.

Este problema foi dividido em duas Fases cada uma subdividida em três Tarefas
de forma a que a resolução das mesmas tivesse uma dificuldade crescente e quem realizava 
o trabalho tivesse de aprofundar cada vez mais os seus conhecimnetos para a resolução dos 
ditos \emph{problemas}, desenvolvendo uma capacidade de interpretação de problemas e resolução dos mesmos 
cada vez maior. 

A primeira fase do projeto centrava-se mais na impressão do mapa com os powerups
e na movimentaçao de um jogador, já a segunda fase era focada no aspeto mais técnico
do jogo onde era necessario fazer os graficos do Jogo, a passagem do tempo e um
boot que conseguisse jogar sozinho.

Neste relatorio, a
Secção ~\ref{sec:problema} descreve o problema a resolver enquanto a
Secção ~\ref{sec:solucao} apresenta e discute a solução proposta pelos
alunos. O relatório termina com conclusões na
Secção ~\ref{sec:conclusao}, onde é também apresentada uma análise
crítica dos resultados obtidos.

\section{Descrição do Problema}
\label{sec:problema}

O \emph{problema} que se pretende resolver no projeto de LI1 é a criação
do jogo Bomberman funcional. Para isso como referido anteriormente o projeto foi 
dividido em duas Fases e cada fase em três Tarefas.

Na primeira Fase o primeiro desafio
era a impressão do mapa do jogo com os PowerUps a partir de uma lista de numeros aleatorios
com base numa \emph{Dimensão} e uma \emph{Seed} de forma a que cada mapa desde que tivesse uma
dimensão ou seed diferente, o mapa produzido seria distinto. ainda na primeira fase outra tarefa 
era a movimentação dos jogadores, ou seja, quando premida uma tecla de movimento, fazer a posição 
do jogador mudar com base na tecla. Na terceira Tarefa foi pedido o \emph{Encode} e o \emph{Decode}
do mapa, ou seja, a compressão do mapa para um formato mais simplificado mas que permitisse depois 
a sua descompressão a partir do codificado.

Na segunda Fase e na primeira Tarefa desta, era necessário fazer a passagem do tempo, sendo
a principal consequência a explosão das \emph{Bombas} e por consequência os elementos do jogo removidos,
ainda nesta Tarefa foi preciso fazer uma espiral que a partir de um determinado momento do jogo calculado 
a partir da dimensão do mapa começa e remove todos os elementos do jogo e substitui por uma pedra fixa 
com o intuito de acelerar a conclusão do jogo. Na segunda Tarefa era necessário a ipressão dos gráficos do
jogo a partir da bliblioteca Gloss. Na terceira Tarefa é necessário a criaçã de um boot capaz de reagir 
e "tomar" decisões por si próprio, por exemplo fugir de uma bomba antes de ela explodir, de modo a 
sobreviver e ganhar a outros boots construidos.


\section{Concepção da Solução}
\label{sec:solucao}

Esta secção deve descrever o trabalho efetivamente desenvolvido pelos
alunos para resolver o problema apresentado na secção
anterior. Segue-se uma sugestão de organização para esta secção.

\subsection{Estruturas de Dados}

A \emph{estrutura de dados} ao longo do projeto foi sempre começada a partir do mapa com os PowerUps, Bombas e jogadores.
Sempre que era necessário algum dado para a realização das Tarefas era percorrido a \emph{Lista de Strings} original
até encontrar a informação relevante.

\subsection{Implementação}

Esta secção deve apresentar as soluções propostas pelos alunos para
atingir os objetivos propostos, assim como a sua \emph{implementação},
devendo também descrever os problemas encontrados e as decisões
tomadas para os ultrapassar. Apesar de poder conter excertos de código
fonte quando relevante, esta secção não deve servir como alternativa à
documentação do código fonte submetido.

\subsection{Testes}

Esta secção deve apresentar alguns dos \emph{testes} definidos pelos
alunos para testar as funcionalidades da solução desenvolvida.

\section{Conclusões}
\label{sec:conclusao}

A secção de \emph{conclusões} resume o restante documento, devendo
também apresentar uma análise crítica dos resultados atingidos tendo
em conta os objetivos definidos.

\end{document}